Very forward region at CepC will be instrumented with a luminometer (LumiCal) designed to enable integral luminosity measurement with a precision of $10^{-3}$ and $10^{-4}$ in \epem collisions at 240~GeV center-of-mass energy and at the $\Zboson_0$ pole, respectively. The precision requirements on the integral luminosity measurement are motivated by the CepC physics program, intended to test the validity scale of the Standard Model through precision measurements in the Higgs and EW sectors.  Many sensitive observables for such measurements, critically depend on the uncertainty of the integral luminosity.
Several technological options for LumiCal design are under study, as described in Sec.\ \ref{sec:lumi_tech}, with emphases on the precision of polar angle and energy reconstruction of Bhabha particles scattered in the t-channel $V (V=\gamma, \Zboson)$ exchange.
Luminometer at CepC is a precision device with challenging requirements on the mechanics and position control. Precision requirements on integral luminosity measurement set the precision of the opening aperture and positioning control of the LumiCal. Various sources of luminosity uncertainty in this respect are reviewed in Sec.\ \ref{sec:lumi_systematics}.
Encouraging estimations on feasibility of the luminosity precision goals are presented. Detailed studies are ongoing, to include the full simulation of physics and machine induced processes and of the detector itself, for various luminometer positioning and technology choices.
