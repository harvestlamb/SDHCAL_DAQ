\section{CEPC Experiment and MC Sample}
\label{sec:CEPC}
\subsection{CEPC experiment}
The CEPC is a future ciruclar electron-positron collider project. Two detectors will be installed at two interaction 
points in the stoarge ring, 50-100 kilometers in circumference. Electrons and positrons collide at each interaction point with center of mass energy 240 - 250 GeV. The luminosity is design to be \ten{2}{34} \lumiunit.
 The higgs boson are produced mainly via associated production with Z boson(96.6\%) as well as much $WW$ fusion (with $\nu_e\bar{\nu}_e H$ in final states,3.06\%) and $ZZ$ fusion(0.29\%, with $e^+e^-H$ in final state).
 
\cite{CEPC_preCDR}. 

\subsection{CEPC detector}
A ILD-like detector is designed as the CEPC detector(CEPC-v1) with additional considerations\cite{CEPC_preCDR}. Detail description of ILD model can be found in \cite{ILD_detector}. All changes need to be implemented into simulation, and iterate with physics analysis and cost estimation.

\subsection{MC Samples}
In this analysis, the signal events are $e^+e^-\to \Zzero \Hboson\to\qqbar + \bbbar/\ccbar/gg$. The standard model background includes di-quark events, di-lepton events, vector boson pair production and higgs production with final states different from the signal. Both background and signal events are generated using Whizard\cite{Wizard_1} with assumption of collision by no-polarization electron-positron, with center of energy of 250 GeV. PHYTHIA 6.4 \cite{PYTHIA64} was used to model the fragmentation and hadronization. The higgs mass was assumed to be 125 GeV and the coupling was set as that predicted by standard model.\par
The generated signal events and backgrounds events with higgs production undergo the GEANT4\cite{Geant4} based detector simulator Mokka\cite{mokka} with CEPC-v1. The simulated hits were digitized and reconstructed by the MarlinReco package. 
%The  LCFIPlus\cite{LCFIPlus} toolkit was utilized to reconstruct the primary vertex, jets and secondary vertex. 
The jets are reconstructed with Durham-like algorithm\cite{Durham} by implementing the toolkit lcfiplus\cite{LCFIPlus}. This toolkit is also capable to reconstruct the primary and secondary vertex, as well as jet flavor tagging.\par
Background events without higgs production undergo fast simulation, which includes:
\begin{itemize}
\item Four momentum of jet(b,c quarks and gluon) is smeared according to a Gaussian function, with jet energy  
resolution $\sigma$ set to be 4\%.
\item Each leptonic track ($e/\mu$) is corrected by momentum resolution and tracking efficiency, whose parameter are obtained from the study of the full simulation.
\item The four momentum of neutrino decaying from the final hadron subtracted from the four momentum of jet.
\end{itemize} 
The fast simulation reconstruct jets using a simplified algorithm. It is faster but with disadvantage that the vertex information is missing from reconstruct. Full simulation sample of the backgrounds are also in production and results will be updated using those samples. Detailed information of MC samples can be found in \cite{Samples}.
\clearpage
