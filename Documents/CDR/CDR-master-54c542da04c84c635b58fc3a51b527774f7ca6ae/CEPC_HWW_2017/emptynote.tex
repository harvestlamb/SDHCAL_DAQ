%\documentclass{cepcnote} 

%\documentclass[usetikz]{cepcnote} % the 'usetikz' option loads tikz.sty in the proper place, 
                                   % avoiding conflicts with graphicx.sty.
                                   % Don't know what tikz.st is? Just ignore this line! :-)

\documentclass[coverpage]{cepcnote} % the 'coverpage' option loads the CEPC Cover Page package 
                                      % ans makes sure that the cover page is generated before the
                                      % note title page. Make sure that the latest version of
                                      % of 'cepccover.sty. is installed on your system!

%\usepackage{graphicx} % This is already loaded by the cepcnote class
                       % Just use it to include your plots!

%\usepackage{cepcphysics} % Contains useful shortcuts. Uncomment to use
                           % See instruction.pdf for details

%%%%%%%%%%%%%%%%%%%%%%%%%%%%%%%%%%%%
%           Title page             % 
%%%%%%%%%%%%%%%%%%%%%%%%%%%%%%%%%%%%

\skipbeforetitle{100pt}

% Title
\title{Measurement of Higgs to WW at CEPC}

% Author
%if not given, typesets ``The CEPC collaboration''
%\author{The CEPC Collaboration}

% if multiple authors/affiliations are needed, use the authblk package
\usepackage{authblk}
\renewcommand\Authands{, } % avoid ``. and'' for last author
\renewcommand\Affilfont{\itshape\small} % affiliation formatting

\author[a]{LIAO Libo}
\author[b]{LI Gang}
\author[b]{RUAN Manqi}

\affil[a]{Hangzhou Normal University}
\affil[b]{Institute of High Energy Physics}

% Date: if not given, uses current date
%\date{\today}

% Draft version: if given, adds draft version on front page, a
% 'DRAFT' box on top of each other page, and line numbers to easy
% commenting. Comment or remove in final version.
\draftversion{1.0}

% Journal: adds a 
\journal{Phys. Lett. B} 

% Abstract
\abstracttext{
  Based on a Monte Carlo sample with planed lumonisity of 5ab^{-1}
}

%%%%%%%%%%%%%%%%%%%%%%%%%%%%%%%%%%%%
%            Content               % 
%%%%%%%%%%%%%%%%%%%%%%%%%%%%%%%%%%%%

\begin{document}

% Test of tikz.sty loadin in cepcnote.cls. Use 'usetkiz' option in documentclass declaration:
% e.g. \documentclass[usetikz]{cepcnote}
% \usetikzlibrary{arrows}

\tikzstyle{int}=[draw, fill=blue!20, minimum size=2em]
\tikzstyle{init} = [pin edge={to-,thin,black}]

\begin{tikzpicture}[node distance=2.5cm,auto,>=latex']
    \node [int, pin={[init]above:$v_0$}] (a) {$\frac{1}{s}$};
    \node (b) [left of=a,node distance=2cm, coordinate] {a};
    \node [int, pin={[init]above:$p_0$}] (c) [right of=a] {$\frac{1}{s}$};
    \node [coordinate] (end) [right of=c, node distance=2cm]{};
    \path[->] (b) edge node {$a$} (a);
    \path[->] (a) edge node {$v$} (c);
    \draw[->] (c) edge node {$p$} (end) ;
\end{tikzpicture}

\section{Introduction}

{\color{red} Place your content here}

\section{Results}

Place your content here

\section{Conclusion}

Place your content here

\end{document}
