% \iffalse meta-comment
%
% Copyright 1993 1994 1995 1996 1997 1998 1999 2000 2001
% The LaTeX3 Project and any individual authors listed elsewhere
% in this file. 
% 
% This file is part of the LaTeX base system.
% -------------------------------------------
% 
% It may be distributed and/or modified under the
% conditions of the LaTeX Project Public License, either version 1.2
% of this license or (at your option) any later version.
% The latest version of this license is in
%    http://www.latex-project.org/lppl.txt
% and version 1.2 or later is part of all distributions of LaTeX 
% version 1999/12/01 or later.
% 
% The list of all files belonging to the LaTeX base distribution is
% given in the file `manifest.txt'. See also `legal.txt' for additional
% information.
% 
% \fi
% Filename: usrguide.tex

%\NeedsTeXFormat{LaTeX2e}[1995/12/01]
\documentclass{article}

\usepackage[UTF8]{ctex}
\usepackage[siunitx]{circuitikz}
\usepackage{tikz}

\title{GEM-Based SDHCAL}

\author{Yu Wang}

\date{\today}


\begin{document}

\maketitle

% \tableofcontents

\section{Readout ASIC}
To read out the GEM-based SDHCAL, we chose an ASIC called MICROROC (MICRO-mesh gaseous structure Read-Out Chip), developed at IN2P3 by OMEGA/LAL and LAPP microelectronics groups. The MICROROC is a 64-channel mixed-signal integrated circuit based on 350nm SiGe technology. Each channel of the MICROROC chip is made of a very low noise fixed gain charge preamplifier optimized for a detector capacitance of 80 pF and able to handle a dynamic range from 1fC to 500fC, two different adjustable shapers (A high gain shaper for small signal and a low gain shaper for large signal), three comparators for tri-threshold readout and a random access memory used as a digital buffer. Other blocks, like 10-bit DAC, configuration register, bandgap voltage reference, LVDS receiver are shared by 64 channels.
\subsection{Analog Part of Microroc}
The analog part of Mi

\section{调研}
调研中发现,窄脉冲大电流LED驱动电路主要以储能电容瞬间放电的方式来完成LED的电流驱动。总结窄脉冲LED电流驱动电路的方式有如下两种:能量压缩、高带宽跨导放大。
\subsection{能量压缩驱动}

\begin{center}
\begin{circuitikz} \draw
 (0,0) to [R=10<\Mohm>] (0,2) -- (0,3)
 to [C=100<\pF>] (1,3)

;\end{circuitikz}
\end{center}
\begin{thebibliography}{1}

\bibitem{A-W:GMS94}
Michel Goossens, Frank Mittelbach and Alexander Samarin.
\newblock {\em The {\LaTeX} Companion}.
\newblock Addison-Wesley, Reading, Massachusetts, 1994.


\bibitem{A-W:GRM97}
Michel Goossens, Sebastian Rahtz and Frank Mittelbach.
\newblock {\em The {\LaTeX} Graphics Companion}.
\newblock Addison-Wesley, Reading, Massachusetts, 1997.


\bibitem{A-W:GR99}
Michel Goossens and Sebastian Rahtz.
\newblock {\em The {\LaTeX} Web Companion}.
\newblock Addison-Wesley, Reading, Massachusetts, 1999.


\bibitem{A-W:DEK91}
Donald~E. Knuth.
\newblock {\em The \TeX book}.
\newblock Addison-Wesley, Reading, Massachusetts, 1986.
\newblock Revised to cover \TeX3, 1991.


\bibitem{A-W:LLa94}
Leslie Lamport.
\newblock {\em {\LaTeX:} A Document Preparation System}.
\newblock Addison-Wesley, Reading, Massachusetts, second edition, 1994.

\end{thebibliography}

\end{document}